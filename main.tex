\chapter{Initiation} % (fold)
\label{cha:cha}

% chapter cha (end)
\section{Definitions} % (fold)
\label{sec:sec}

% section sec (end)

\begin{example}
	$l^2(\N)=\{n\in \mathbb{N}\mapsto f(n)\in\C$ t.q. $\sum_{n\geq 0}|f(n)|^2 < \infty\}$
	
	$l^2(\N)$ est $\C$ espace. $\forall f, g\in l^2(\N):$ $$ (f|g)_{l^2(\N)}\overset{\text{def}}{=\joinrel=}∑_{n≥0}f(n)\overline{g(n)}.$$

Soit $(f_n)_{n\in\N}$ une suite de Cauchy dans $l^2(\N)$: 
\[\forall \eps >0\ \exists N\in \N\ \forall n>p\geq N:\quad  ||f_n-f_p||_{l^2(\N)}<\eps. \label{eqn:cauchy-suite}\tag{\textasteriskcentered}\]
	
\textbf{Question.} $\exists f\in l^2(\N)$ telle que $\lim\limits_{n\to ∞}f_n=f$?

\eqref{eqn:cauchy-suite} $\Leftrightarrow$ $\forall \eps >0\ \exists N\ t.q.\ \forall n>p\geq N\ ||f_n-f_p||^2=\sum\limits_{j\geq 0}|f_n(j)-f_p(j)|^2\leq \eps^2$\\
$\Rightarrow \quad |f_n(j)-f_p(j)|\leq \eps\ \forall j\in \N$.\\
$\Rightarrow \forall j\in \N\ (f_n(j)))_{n\in \N}$ est de Cauchy dans $\C$ qui est complet, donc $\exists f(j)\in \C$ telle que $\lim\limits_{n\to \infty} |f_n(j)-f(j)|=0$.

Il faut montrer que $f$ est la limite dans $l^2(\N)$ de la suite $f_n$.\\
$\forall \eps>0\ \exists N$ t.q. $\forall n>p\geq N \sum\limits_{j\geq 0}|f_n(j)-f_p(j)|^2\leq \eps^2$\\
$\Rightarrow$ $\forall J\in\N\ \underbrace{\sum\limits_{j=0}^J |f_n(j)-f_p(j)|^2}_{\text{somme partielle}}\leq \eps^2$, par passage à la limite sur $p$: $\sum_{j=0}^J|f_n(j)-f(j)|^2\leq \eps^2$

Conclusion: $\forall \eps>0\ \exists N$ telle que $\forall n\geq N\ ||f_n-f||<\eps \Longrightarrow \lim\limits_{n\to ∞}f_n=f$.	

Mais $f\overset{\text?}\in l^2(\N)$.

Vérifions que $f\in l^2(\N)$:\\$( \sum_{j\geq 0}|f(j)|^2 )^{1/2}=(\sum_{j\geq 0} |f_n(j)-f(j)+f(j)|^2)^\frac 12=||\underbrace{f-f_n}_{\in l2n }+\underbrace{f_n}_{\in l2n}||\leq ||f-f_n||+||f_n||<+∞$.
\end{example}



\begin{theorem}[Projection orthogonale]
	Soit $H$ un espace de Hilbert et $C$ une partie \texttt{convexe} \texttt{fermée} et \texttt{non vide} de $H$. Alors $\forall x\in H\ \exists ! y_0\in C$ t.q.
	\begin{enumerate}
		\item $\dist(x, C):=\inf\{d(x,y), y\in C\}=\inf\{||x-y||_H, y\in C\} = ||x-y_0||_H$
		\item $\forall y\in C\ \Re(x-y_0| y-y_0)\leq 0$ !?
	\end{enumerate} 
	$y_0$ est la projection orthogonale de $x$ sur $C$.
	
\end{theorem}


\begin{remark}
	\leavevmode
	\begin{enumerate}
		\item $C$ est convexe si $\forall x, y \in C$ $[x,y]=\{tx+(1-t)y, t\in[0,1]\}\in C$
		% img convexity
		\item $H=\R^2:\ [x,y]\in C$
		\item si $x_0\in C$ dans le cas $y_0=x_0$ et $\dist(x_0, C)=0=||x_0-x_0||_H$
	\end{enumerate}
\end{remark}





\begin{proof}
	Notons par $d=d(x,C)>0\ (x\in H\diagdown C)$. Soit $y, z\in C$ on pose $b=x-\frac12(y+z),\ c=\frac12(y-z):\ ||b||=||x-\frac12\underbrace{(y+z)}_{\in C}||\geq d$. On a aussi $b-c=x-y$ et $b+c=x-z$ $\Rightarrow ||x-y||^2+||x-z||^2=||b-c||^2+||b+c||^2=(b-c| b-c)+(b+c|b+c)=||b||^2+||c||^2-(b|c)-(c|b)+||b||^2+||c||^2 + (b|c)+(c|b)$.
	
	$||x-y||^2+||x-z||^2=2(||b||^2+||c||^2)\geq 2 d^2+2\frac14||y-z||^2 \Rightarrow ||y-z||^2\leq 2(||x-y||^2-d^2)+2(||x-z||^2-d^2)$. Pour $n\in N$ $C_n=\{y\in C ||x-y||^2\leq d^2+\frac1n\}$ est fermée dans H (boule fermée).
	
	Puisque $C$ est fermé, $C_n=\{y\in H ||x-y||^2\leq d^2+\frac1n\}\cap C$ est fermé dans $C$.
	De plus: $\delta (n):=sup\{||y-z||, (y,z)\in C_n\times C_n\}\leq * sup\{[2(||x-y||^2-d^2)+2(||x-z||^2-d^2)]^\frac12, y,z\in C_n$ $\Rightarrow$ $\delta (n)\leq \frac2{n^\frac12}\to 0$ quand $n\to +\infty$.

$H$ est complet et $C\subset H_x$ $c$ est fermé. $C$ est un espace métrique complet. Il satisfait le critère de Cantor: $\bigcap\limits_n C_n=\{y_0\}$.


$y_0\in\cup_n C_n\ d^2\leq ||x-y_0||^2\leq d^2+\frac1n\ \forall n\in\N*=\N\\\{0\}$
$\Rightarrow ||x-y_0||=d^2$.

Aadff ii): $\forall t\in[0,1],\ \forall\in H\ \phi(t)=||\underbrace{y_0+t(y-y_0)}_{\in C}-x||^2 = ||y_0-x||^2+2tRe(y_0-x|y-y_0)+t^2||y-y_0||^2$. $\phi(0)=d^2\leq \phi(t)\ \forall t\in(0,1]$ $\Rightarrow \phi'(0)\geq 0$. $\phi'(t)=2Re(y_0-x|y-y_0)+2t||y-y_0||^2$. $\phi'(0)\leq 0 \Rightarrow 2Re(y_0-x|y-y_0)\leq 0\Rightarrow (i)$.

\end{proof}


\begin{theorem}[corollaire]
	Soit $F$ un sous-espace \textsc{fermé} de $H$ alors: $H=F\bigoplus F^\perp$.
\end{theorem}
\begin{proof}
	\begin{itemize}
		\item $F$ est convexe puisque $\forall \alpha,\beta\in\C \forall x, y\in F\ \alpha x+\beta y\in F$ $\Rightarrow$ Celaeitvnai.. $\alpha = t,\ \beta=1-t\ t\in[0,1]$.
	
	On peut lu applieuer le Thm 1:
		\item On a tanga.. $F+F^\perp \subset H$ et $F+F^\perp = F\bigoplus F^\perp$ onu si $x\in F\cap F^\perp$ $\Rightarrow$ $(x|x)=0=||x||^2$ $\Rightarrow$ $x=0_H$
		
		Soit $x\in H$, et $y_0\in F$ sa projectien an Thegerale: $\forall d\in \C, y\in F, y_0+dy\in F$ et donc $Re(x-y_0| y_0+dy-y_0)\leq 0$ $\Rightarrow$ $Re(x-y_0|dy)\leq 0$
		
		$d=(x-y_0|y)$ $\Rightarrow$ $(x-y_0)$
		...
		
	Conclusion $Re(x-y_0|dy)$.. donc $H=F\bigoplus F^\perp$.
	\end{itemize}
\end{proof}

\begin{definition}
	Dans ces condition, l'application $P:x\in H, x=x_1+x_2, x_1\in F, x_2\in F^\perp \overset{P}{\mapsto} x_1\in F$. est le \textsc{Projection Orthogonal} sur $F$.
\end{definition}

\begin{examplebox}
	Montrer que P est linéaire continue et satisfait $P^2=P$.
\end{examplebox}

\begin{definition}
	Une partie $A$ de $H$ est dite \textsc{Totale} si le plus petit sous espace fermé contenant $A$ et $H$.

	$H$ est \textsc{Séparable} si $H$ admet une famille totale dénombrable.
\end{definition}

\begin{example}
	$H=l^2(\N): \mathcal{F}=\{e_0, e_1, ...\}$ avec $e_j(i)=\delta_{ij}\to (0,0,..., 0,1,0,... 0)$. $\mathcal{F}$ est totale. Elle est dénombrable, $l2n$ est séparable.
\end{example}

\begin{theorem}
	Soit $H$ un espace de Hilbert et $A\subset H$:
	\begin{enumerate}
		\item $\overline{vect(A)}=(A^\perp)^\perp$
		\item $A$ est on sous-espace alors $(A^\perp)^\perp=\bar A$
		\item $A$ est totale $\Leftrightarrow$ $A^\perp=\{0_H\}$
	\end{enumerate}
\end{theorem}

\section{Séries dans un espace vectoriel normé} % (fold)

Soit $(E, ||\cdot||_E)$ un \texttt{espace vectoriel normé} (e.v.n).
\begin{definition}
	On appelle \textsc{Série} de terme général $u_n\in E$ la suite $(S_N)_{N\in \N}$ de $E$ t.q.  $S_N=\sum\limits_{n=0}^Nu_n$.
	La série est \textsc{Convergente} dans $(E, ||\cdot||_E)$ si le suite $(S_N)_{N\in\N}$ admet une limite dans $E$: $S$ --- toute la somme de la somme la série.
\end{definition}



\begin{definition}
	Une série $\sum u_n$ est dite \textsc{Absolument Convergente} (AC) si la série $\sum ||u_n||_E$ est convergente dans $\R^+$.
\end{definition}

\begin{theorem}
	Si $E$ est \texttt{complet} (espace de Banach/Hilbert) Alors toute série AC est convergente et $||\sum\limits_{n=0}^\infty||\leq \sum\limits_{n=0}^\infty||u_n||$. !?
\end{theorem}
\begin{proof}
	$J_n=\sum\limits_{n=0}^N||u_n||$ et convergente $\Leftrightarrow$ $(J_n)_{N\in\N}$ est de Cauchy $\forall \eps >0\ \exists K\ t.q.\ \forall N>P\geq K\Rightarrow |J_n-J_p|\leq \eps$. $\sum_{j=p+1}^N||u_j||\leq \eps$. meus $||S_n-S_p||=||\sum_{j=p+1}^Nu_j||\leq\sum_{j=p+1}^N ||u_j||$ Tnegalite trianguler.
	
	$\Rightarrow N>p\leq K\Rightarrow ||S_N-S_P||\leq \eps \Leftrightarrow (S_N)_{N\in \N}$ est de Cauchy dans $E$ et donc convergente.
	
	D'au the peut $||S_n||=||\sum_{j=0}^n u_j||\leq\sum_{j=0}^n\leq \sum_{j=0}||u_j||$ $\Rightarrow\ ||\sum_{j=0}u_j||\leq \sum_{j=0}||u_j||$. Cqfd.
\end{proof}

\begin{definition}
	Une suite $(x_n)_{n\in\N}{n\in \N}$ de H est dite \textsc{Orthogonal} si $(x_i|x_j)=0\ \forall i≠j$.
\end{definition}

\begin{theorem}
	Soit $(a_n)_{n\in \Z}$ une suite orthogonal dans un espace de Hilbert $H$. Alors le série $∑x_n$ est convergente $\Longleftrightarrow$ $∑_{n≥0}||x_n||_H^2$ est convergente et \[ ||∑_{n≥0}x_n||^2_H=∑_{n≥0}||x_n||_H^2.\]
\end{theorem}
\begin{proof}
	$\forall$ $l>p$ on a $||∑_{n=l}^p||^2=(∑_n=e^p x_n | ∑_n=e^p x_n)=∑_n,n'=l(x_n|x_n')=∑_n=l^p||x_n||^2$ Alors $(x_n)_{n\in\N}{n\in \N}$ est de Cauchy $\Leftrightarrow$ $(||x_n||^2)_{n\in \N}$ est de Couchy dans $\R$.

D'aute peut $S_N=∑_{n≥0}^N x_n$ $\Rightarrow$  $||S_N||^2=∑_{n≥0}^N||x_n||^2$.
Alors $S=lim S_N=∑x_n$ $||S||^2=||lim N S_N||^2=lim ||S_N||^2$ par continite de la $||•||$ et donc $||S||^2=lim_N∑_n≥0^N||x_n||^2=∑_{n≥0}||x_n||^2$
\end{proof}

\section{Bases Hilbertiennes} % section #3

\begin{definition}
	On appelle \textsc{Base Hilbertienne}, une suite de vecteur $(x_n)_{n\in\N}{n\in \N}$ telle que 
	\begin{enumerate}
		\item $\forall n, m (x_n|x_m)=δ_{nm}$,
		\item $\vect\{(x_n)_{n\in\N}n\in\N\}=H$ $\Leftrightarrow$ $\vect{(x_n)_{n\in\N}{n\in \N}}^\perp=\{0_H\}$ $\Leftrightarrow$ $(x_n)_{n\in\N}{n\in \N}$ est totale.
	\end{enumerate}
\end{definition}

\begin{theorem}[Inégalité de Bessel]
	Soit $(x_n)$ une suite \texttt{orthonormale} ($\forall n, m (x_n|x_m)=δ_{nm}$) dans $H$. Alors $\forall x\in H ∑_{n≥0}|(x|x_n)|^2$ est convergente et $∑_{n≥0}|(x|x_n)|^2≤||x||^2$.
\end{theorem}

\textbf{Exemple:} $H=l^2(\N)$. $(e_n|e_m)=∑_{k≥0}e_n(k)\overline{e_m(k)}=∑_{k≥0}δ_{nk}δ_{mk}=δ_{nm}$. En fait on montre que $∑_{n≥0}|(e_n|x)|^2=||x||^2$ c'est une base Hilbertienne.
\begin{proof}
	Sait $x\in H$ on pose $y_i=(x|e_i)e_i$ et $Y_N=∑_1^Ny_i$, $Z_N=X-Y_N$. Alors: $(Z_N|y_i)=(X-Y_N|y_i)=(X|y_i)-(Y_N|y_i)$. $(x|y)=(x|(x|e_i)e_i)=\overline{(x|e_i)}(x|e_i)=|(x|e_j)|^2$. $(Y_N|y_i)=∑_{j=1}^N(y_j|y_i)$ mais $y_j\perp y_i$ $\Rightarrow$  $(Y_N|y_i) =||y_i||^2$ si $N≥i$.
	(autrement =0)

Dans ces conditions puisque $||y_i||^2=|(x|e_i)|^2$. Alors $(Z_n|y_i)=0$ $\Rightarrow$  $(Z_N|Y_N)=0$ cas $Y_n=∑_{i=0}^Ny_i$ $\Rightarrow$  $||x||^2=||Z_n||^2+||Z_N||^2$ $(x=Zn+Yn et Z_n\perp Y_n)$
$\Rightarrow$  $||y_n||^2=∑||y_n||^2≤||x||^2$

La seuie $∑^N||y_n||^2$ est positive, majorée donc convergente et par passage à la limite: $∑_{n≥0}||y_n||^2=∑|(x|e_n)|^2≤||x||^2$. QED

\end{proof}


\begin{theorem}[Egalité de Parseval]
	Soit $(e_n)$ une base Hilbertienne de $H$ alors 
	\begin{enumerate}
		\item La série $∑_{n≥0}|(x|e_n)|^2$ est convergente et $||X||^2=∑_{n≥0}|(x|e_n)|^2$,
		\item Ls série $∑_{n≥0}(x|e_i)e_i$ est convergente dans $H$ et $∑_{i≥0}(x|e_i)e_i=x$.
	\end{enumerate}
\end{theorem}
\begin{proof}
	En utilisant le théorème précédent alors $∑|(x|e_i)|^2$ est convergent on utilise l'identité de la médiane: $∑(x|e_i)e_i$ et convergente dans $H$ $(||(x|e_i)e_i||^2 =|(x|e_i)|^2)$.
	On pose $y=∑_{i≥0}(x|e_i)e_i$ alors $||y||^2 =∑_{i≥0}|(x|e_i)|^2)$ mais $(y|e_j)=(∑(x|ei)ei|e_j)=∑(x|e_i)(e_i|e_j)=(x|e_j)$ ...
	Conclusion $\forall j\in \N (x|e_j)=(y|e_j)$ $\Leftrightarrow$
	$ (x-y|e_j)=0$ $\Rightarrow$  $x-y\in\vect((e_n)_{n\in\N})^\perp$
	$\Rightarrow$  $x-y=0_H$ $\Leftrightarrow$ $x=y=∑(x|e_i) e_i ||x||^2=∑_{i≥0}|(x|e_i)|^2$
\end{proof}
\begin{remark}
	Si $(e_n)_{n\in \N}$ est une suite orthonormal telle que $\forall x\in H x=∑_{i≥0}(x|e_i)e_i:\ x=\lim_N ∑_{i≥0}^N a_ie_i$ où $a_i=(x|e_i)\in\C$ 
	
	
	$\in \vect\{(e_n)_n\in \N\}; a_i=(x|e_i)$ $\Rightarrow$  $\vect\{(e_n)_n\in\N\}=H$. $(e_n)_n\in \N$ est une base Hilbertienne.
	ii)>> $(e_n)_n\in\N$ est base Hilbertienne de $H$ $\Leftrightarrow$ $\forall x\in H:\ ∑(x|e_i)e_i=x $
	$∑(x|e_i)e_i=x$ $\Leftrightarrow$ $∑|(x|e_i)|^2=||x||^2 $i $>>$ $(e_n)$ est une base Hilbertienne de $H$ $\Leftrightarrow$ $∑|(x|e_i)|^2=||x||^2 \forall x\in H$
\end{remark}


Exemple (suite):
$H=l^2(\N)$. $(e_n)_{n\in \N} t.q. e_n(k) =δ_{nk}$.

$u\in H$ $\Leftrightarrow$ $∑_{n≥0} |u(n)|^2=||u||^2$ mais $u(n)=(u|e_n)=∑u(k)e_n(k)$ $\Leftrightarrow$ $∑_n≥0 |(u|e_n)|^2=||u||^2$, $\Rightarrow$  c'est une base Hilbertienne. !?

\section{Dual d'un espace de Hilbert} % (fold)

On rappelle que si $S$ est un e.v.n. une \textsc{Forme Linière} sur $X$ --- une application linière de $X$ dans $\C$ soit $l: X \rightarrow  \C:\ \forall d \in \C\ \forall x, y\in X l(x+dy)=l(x)+dl(y)$. L'ensemble des formes linéaires de $X$: est un espace vectoriel $X^*$. On considère $X'$ dual topologique: c'est l'espace vectoriel des formes linéaires continues sur $X:\ \{l:(X,||•||_X)\rightarrow (\C, |•|)\}$.

\begin{exercise}	
	$l$ est continue $\Leftrightarrow$ 
	\[\exists C>0\ x \forall x\in X, |l(x)|≤C||x||\label{eq:cont} \tag{\textasteriskcentered}\]
\end{exercise}

On définit $l\in X'$, $||l||=\inf\{C>0 \text{ t.q. \eqref{eq:cont} est satisfait}\} = \sup\{ |l(x)|\ |\ ||x||=1\}$.
$(X', ||•||)$ est un espace de Banach (un e.v.n. \texttt{complet})



\begin{theorem}[Théorème de représentation de Riez]. Soit $H$ est un espace de Hilbert $H'$ son dual topologique. On définit $I :H\rightarrow H"$ par $\forall x\in H I(x)=(•|x)$. Alors $I$ set un isomorphisme isométrique de $H\rightarrow H'$.
\end{theorem}

\begin{remark}
	$H=\C^n$, une forme linéaire sur $\C^n$: $l$. 
	$l(x_1,...\, ,x_n)=∑_{i=1}^n a_ix_i,\ a_i\in \C$
	$|l(x)|=|∑_{i=1}^na_ix_i|≤sup\{a_i|\}•||x||_{\R^n}$. Ici $X^*=X'$ !?

	$$l(x)=(a_1,a_2,...\,, a_n)\mqty({x_1\\ x_2\\ \vdots\\ x_n})$$
	$=(\bar a|x) \forall x\in \C^n$
	$\forall l\in X',\ \exists a\in \C:\quad l(x)=(x|\bar a)$
	Généralisation à la dimension quelconque c'est le théorème de Riez:
	$\forall l\in H'\ \exists a\in H\  \forall x\in H:\ l(x)=(x|a)|$
\end{remark}
 

\begin{proof}
	
	
	Soit $l\in H'$ $l≠0_h'$ $\Leftrightarrow$ 
	
\ifcomment
	lei $l≠H$ pueque $\exists \in tq l(x)≠0_h$ On Satit que $\ker l$ est ferme sait $(x_n)_{n\in\N}n\in|N$ une suiti de kei $l$ convergente dans $H$: $x_n\rightarrow x\in H$ mais latren time: $l(x_n)\rightarrow  l(l(x)$ mais l(x_n)=0 \forall n. $\Rightarrow$  l(x)=0 x\in ker Alors H=ker l \oplus (ker l )^perp (thm propilere
	puisque (ker l)≠H $\Rightarrow$  \ker lY\perp–\{0--_h]) Soit x\in ker \phi ^\perp, ||x||=1|)| x≠0_H
	
	\forall y\in H soit z=-l(x)y_l)y)x\in H et l(lx)=-l(x_l(y)+l(y)l(x)=0 x\in rerl $\Rightarrow$  (x|z)=0
	
	$\Rightarrow$  )x|-l(x)y+l(y)x) $\Rightarrow$  l(x) \rightarrow  l(x)(y|x)=l(y)(X|X) $\Rightarrow$  \forall y\in H l(y)=(y|\overline{l(x)X)|)|))
	
	$\forall l\in H' \exists a\in H$ t.q. $\forall x\in H l(x)=(x|a)|$ I est surjective. Montres que I est injective.
	Soit $x\in H$ t.q. $I(x)=O_H'$ $\Leftrightarrow$ $\forall y\in H I(x)(y)=(y|x)=0$ $\Rightarrow$  $x\perp H$ $\Rightarrow$  $X=0_H$ $\ker I=\{O_h\}$ $I$ est injective donc bijection.


En fin: $||I(x)||=\sup\P|(y|x)|, ||y||=1\} -||x|| isométrie.)|$

Parce que $|(y|x)|≤||y||$ $||x||=||x||$ $y=\frac x{||x||}$ $||y||=1$ $|(y|x)|=||x||$
\fi

\end{proof}
\begin{remark}
	Si l est anti-linéaire: $\forall d\in \C\ \forall x,y\in H\ l(x+dy)=l(x)+\bar d l(y)$ et $\exists u $ t.q. $\forall x\in H:\  l(x)=(u|x)$
\end{remark}

\section{Convergence faible dans les espaces de Hilbert} % (fold)

\subsection{Définition et premières propriétés} % (fold)
\label{sub:definition_et_premieres_proprietes}

\begin{definition}
	Soit $H$ un espace de Hilbert. Une suit$ (x_n)_{n\in\N}{n\in \N}$ de $H$ est dit \textsc{Converge Faiblement vers} $X\in H$ si $\forall y\in H (x_n|y)
ightarrow (x|y)$. On notera $x_n\rightharpoonup x$, $x$ est dite limite faible de $(x_n)_{n\in\N}{n\in \N}$.
\end{definition}
Exp. $H=l^2(\N)$, $x_n\in l^2(\N^*)$ t.q. $x_n(j)=δ_{nj}$.

$(x_n)_{n\in\N}n\in\N$ est une base hilbertienne de H. On regarde la convergence faible. Soit $y\in l^2(\N^*)$ on doit calculer $\lim_{n\to +∞}(x_n|y)$, $(x_n|y)=∑_j x_n(j)\overline{y(j)}=\overline{y(n)}$. $|(x_n|y)|≤|y(n)|$ on sait $∑_j|y(j)|^2<+∞$ $\Rightarrow$ $|y(j)|\to 0$ qd $j\to+∞$ et donc $|(x_n|y)|=|y(n)|\to 0$ qd $n\to +∞$. On ercit $0=(0_H|y)$ alors $\lim_n(x_n|y)=(0_H|y)$. $0_H$ est une limite de la suite $(x_n)_{n\in\N}{n\in \N}$ (On montrera la limite faible est unique).
$\norm{x_n}^2=∑_j|x_n(j)|^2=1$ $\Rightarrow$ $x_n\not\to 0$ puisque $\lim_n\norm{x_n-0_H}=\lim_n\norm{x_n}=1\not\to 0$. $0_H$ n'est pas limite de la suite $(x_n)_{n\in\N}$.

\begin{proposition}
	La limite faible, si elle existe elle est unique.
\end{proposition}
\begin{proof}
	Supposons que $\forall y\in H (x_n|y)\to (x|y)$ et $(x_n|y)\to (x'|y),\ x,x'\in H$. Supposons $x\neq x'$ $\Leftrightarrow$ $x-x'≠0_H$ $\Rightarrow$ $\exists y\in H$ t.q. $(x|y)≠(x'|y)$ (*)
		\begin{remark}
			On suppose (*) faux: $\forall y\in H (x|y)=(x'|y)$ $\Leftrightarrow$ $(x-x'|y)=0$ $\Rightarrow$ $x-x'\perp H$ $\Rightarrow$ $x-x'=0_H$ c'est Absurde.
		\end{remark}
	On pose $u_n=(x_n|y)$ $u=(x|y)$ $u'=(x'|y)$
	$u_n\to u:\ \forall  ε>0\ \exists N$ t.q. $\forall n≥N |u_n-u|≤ε$. On choisit $ε<|u-u'|$ alors on a toujours si $n≥N$ $|u_n-u'|=|u_n-u+u-u'|=||u-u'|-|u_n-u|| ≥|u-u'|-ε≥\frac{|u-u'|}2$ $\Rightarrow$ $\forall n≥N |u_n-u'|≥\frac{|u-u'|}{2}$ $\Rightarrow$ $|u_n-u'|\not\to 0$ $\Leftrightarrow$ $u_n\not\to u'$ QED.
\end{proof}
Dans l'exemple précédent $0_H$ est la limite unique de la suite $(x_n)_{n\in\N}$

Exemple. $H=L^2(\R)$. Soit $H_0\in C^∞_c(\R)$ On pose $\forall n\in \N$, $φ_n(x)=φ_0(x-n)\ x\in\R$.

\begin{rappel}
	
	$C_c^∞(\R)$ ensemble des fonctions $f:\R\mapsto  \C$. \\
	* support $f$ compact : borne et ferme.\\
	* $f\in C^n_(\R)$
	$\Leftrightarrow$ $f\in C_X^∞(\R)$
	support $f=\overline{\{x\in \R, f(x)≠0\}}$
	
	$L^2(\R)=\overline{C_x^∞(\R)}|_{\norm{•}_{L^2(\R)}}$
\end{rappel}

$φ_0\in C_C^∞(\R)$, $\forall n\in\N φ_n(x)=φ_0(x-n)$.

$\forall \psi\in L^2(\R)$: $(φ_n|\psi) \to 0=(0_H|\psi)$ $(φ_n|ψ) = ∫_\R \dd{x} φ_n(x) \overline{ψ(x)} = ∫_{n-1}^{n+1} \dd{x} φ_0(x-n)\bar ψ(x)$.  $|(.|.)|_{L^2((n-1,n+1))}≤\norm{•}\norm{•}$ $\Rightarrow$  $∫_{n-1}^{n+1} |φ_0(x-n)|^2 \dd{x} = ∫_{-1}^{+1} |φ_0(t)|^2 \dd{t} =1$ $\Rightarrow$ $|(φ_n|ψ)|≤(∫_{n-1}^{n+1}|ψ(x)|^2\dd{x})^{\frac 12}$

$ψ\in L^2(\R)$ $\Rightarrow$ $∫_{n-1}^{n+1} |ψ(t)|^2 \dd{t} \to 0$ quand $n\to +∞$. $\norm{ψ}=∑_n∫_{n-1}^{n+1}$ $|ψ|^2\dd{t}<∞$.

\begin{proposition}
	\begin{enumerate}
		\item soit $(x_n)_{n\in\N}$ t.q. $x_n\rightharpoonup x \in H $alors $(x_{k(n))})_{n\in\N}$ Converge faiblement et $x_{k(n)}\rightharpoonup x$
		\item si $(x_n)_n\in\N$ et $(y_n)_{n\in\N}$ sait deux suites t.q. $x_n\rightharpoonup x$ et $y_n \rightharpoonup y$ alors $x_n+y_n\rightharpoonup x+y$
		\item si $x_n\rightharpoonup x$ et soit $(d_n)_{n\in\N}$ une suite des $\C$ t.q. $d_n\to d \in \C$ $\Rightarrow$ $d_nx_n\rightharpoonup dx$.
	\end{enumerate}
\end{proposition}
\begin{proof}
	\begin{enumerate}
		\item i est évident $\forall y\in H$ si $u_n=(y|x_n)$ $\Rightarrow$ $u=(y|x)$ $\Rightarrow$ $u_{k(n)}\to u$ $\Rightarrow$ i)
		\item $\forall y\in H (y|x_n+z_n)=(y|x_n)+ (y|x_n) \to (y|x)+(y|z)=(y|x+z)$.
		\item On suppose $\forall y\in H (x_n|y)\to (x|y)$ et $d_n\to d$.
		$(d_nx_n-dx|y)=(d_nx_n-dx_n+dx_n-dx|y)=(d_n-d)(x_n|y)+d(x_n-x|y)$ $\Rightarrow$ $|(d_nx_n-dx|y)|≤|d_n-d||(x_n|y)|+|d||(x_n-x|y)|$
		\begin{enumerate}
			\item $(x_n|y)\to (x|y)$ $\Rightarrow$ $\exists M$ t.q. $|(x_n|y)|≤M\ \forall n\in\N$ $\Rightarrow$ $|d_n-d||(x_n|x)|≤|d_n-d|M\to 0 qd n\to +∞$. 
			$|(x_n-x|y)|\to 0 qd n\to +∞$ par (*) la proposition est démontrer.
		\end{enumerate}
	\end{enumerate}
\end{proof}
\begin{remark}
	On a toujours que $|(x_n-x|y)|≤\norm{x_n-x}_H\norm{y}_H$. Si $\lim_n\norm{x_n-x}=0$ $\Leftrightarrow$ $\lim_n x_n=x$ $\Rightarrow$ $x_n\rightharpoonup x$
	! l'inverse est faux en général.
\end{remark}
\begin{proposition}
	Si $x_n\rightharpoonup x$ dans $H$ alors $\lim_{n\to + ∞}\inf\norm{x_n}≥\norm{x}$.
\end{proposition}
\begin{remark}
	Si $(x_n)_{n\in\N}$ converge $\exists x\in H$ et $\lim_{n\to +∞}\norm{x_n-x}=0$ alors par $|\norm{x}-\norm{x_n}|≤\norm{x-x_n}$ $\Rightarrow$ $\lim_{n\to ∞}\norm{x_n}=\norm{x}$.
	Mais si on a que $x_n\rightharpoonup x$ on ne sait pas que la suite $\norm{x_n}$ converge, c.a.d. que la limite existe par contre $\lim_n\inf\norm{x_n}$ = $\lim_{n\to ∞}\inf\{\norm{x_k}, k≥n\}$ et $\lim_n\sup\norm{x_n}-\lim_{n\to +∞} \sup\{\norm{x_k}, k≥n\}$ existe toujours.
\end{remark}
\begin{proof}
	Puisque $x_n\rightharpoonup x$, alors $(x_n|x)\to (x|x)=\norm{x}^2$ en utilisant Cauchy Schwartz $|(x_n|x)|≤\norm{x_n}{x}$.
	$\Rightarrow$ $\norm{x}^2≤\norm{x_n}\norm{x}$ $\Leftrightarrow$ $\norm{x}≤\norm{x_n}$ $\Rightarrow$ $\norm{x}≤\lim_{n\to∞}\inf\norm{x_n}$.
\end{proof}
\begin{proposition}
	Soit $(x_n)_{n\in\N}$ une suite dans $H$. Alors 
		$x_n\to x$ $\Leftrightarrow$ $x_n\rightharpoonup x$ et $\lim_n\sup\norm{x_n}≤\norm{x}$
\end{proposition}
\begin{proof}
	($\Rightarrow$) $x_n\to x$ $\Rightarrow$ $x_n\rightharpoonup x_n$ et $\norm{x_n}\to \norm{x}$
	($\Leftarrow$) $\norm{x-x_n}^2 = \norm{x}^2+\norm{x_n}^2 - 2\Re (x|x_n)$
	$\lim_n\sup \norm{x-x_n}^2≤ \norm{x}^2+\lim_n\sup\norm{x_n}^2 - 2\norm{x}^2$.
	$\lim_n\sup \norm{x-x_n}^2≤\lim_n\sup \norm{x_n}^2-\norm{x}^2 ≤0$
	$\Rightarrow$ $\lim_n\sup \norm{x-x_n}^2=0 ≥\lim_n\inf \norm{x-x_n}^2≥0$
	$\Rightarrow$ $\lim_n\sup \norm{x-x_n}^2 =\lim_n\inf\norm{x-x_n}^2=\lim_n\norm{x}$
\end{proof}

\begin{example}
	Soit $(x_n)_{n\in\N}$ une suite bornée de H. Soit $D\subset H$ dense ($\bar D=H$). Alors $x_n\rightharpoonup x$ sur $H$ $\Leftrightarrow$ $(x_n|y)\to (x|y)\ \forall y\in D$.
\end{example}