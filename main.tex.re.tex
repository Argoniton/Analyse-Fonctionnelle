\chapter{Initiation} % (fold)

% chapter cha (end)
\section{Les espaces de Hilbert} % (fold)
$\mathbb{K}=\C$ ou $\mathbb{K}=\R$.
\begin{definition}
	Soit $E$ un $\mathbb{K}$ espace vectoriel. Une application $φ:E\times E \rightarrow  \mathbb{K}$ est une \textsc{Forme Hermitienne}
	\begin{enumerate}
		\item $\forall y\in E$: $φ(•,y):E\rightarrow \R$ est linéaire
		\item $\forall(x,y)\in E\times E$: $φ(x,y)=\overline{φ(y,x)}$
	\end{enumerate}
\end{definition}

\begin{definition}
	Un \textsc{Produit Scalaire} est une forme hermitienne \texttt{définie positive}: $\forall e\in E\ φ(x,x)≥0$; $φ(x,x)=0$ $\Leftrightarrow$ $x=0_E$. Notation:
		$$φ(x,y):=(x|y)$$
\end{definition}
\begin{definition}
	Le couple $(E,(•|•))$ s'appelle un \textsc{Espace Préhilbertien}.
\end{definition}
\begin{definition}
	On définit la \textsc{Norme} sur $E$: $\forall x\in E\ \norm{x}_E=(x|x)^{\frac12}$.
\end{definition}
\begin{remark}
	En particulier on a l'inégalité de Cauchy-Schwartz:
	$$\forall(x,y)\in E^2\ |(x|y)|≤\norm{x}\norm{y}.$$
	Donc inégalité triangulaire. Ainsi c'est vraiment une norme. 
\end{remark}

\begin{definition}
	$x,y\in E$ sont dits \textsc{Orthogonaux} si $(x|y)=0$. Nous dénotons cela comme $x\perp y$.
\end{definition}

\begin{definition}
	$(E, \norm{•})$ est dit \textsc{Complet} si toutes les suites de Cauchy de $E$ convergent dans $E$.
\end{definition}
\begin{definition}
	\textsc{Une Espace de Hilbert} est un espace préhilbertien complet pour la distance $\norm{•-•}=(•-•|•-•)^{\frac12}$.
\end{definition}

\begin{example}
	$l^2(\N)=\{n\in \mathbb{N}\mapsto f(n)\in\C$ t.q. $\sum_{n\geq 0}|f(n)|^2 < \infty\}$
	
	$l^2(\N)$ est $\C$ espace. $\forall f, g\in l^2(\N):$ $$ (f|g)_{l^2(\N)}\overset{\text{def}}{=\joinrel=}∑_{n≥0}f(n)\overline{g(n)}.$$

Soit $(f_n)_{n\in\N}$ une suite de Cauchy dans $l^2(\N)$: 
\[\forall \eps >0\ \exists N\in \N\ \forall n>p\geq N:\quad  ||f_n-f_p||_{l^2(\N)}<\eps. \label{eqn:cauchy-suite}\tag{\textasteriskcentered}\]
	
\textbf{Question.} $\exists f\in l^2(\N)$ telle que $\lim\limits_{n\to ∞}f_n=f$?

\eqref{eqn:cauchy-suite} $\Leftrightarrow$ $\forall \eps >0\ \exists N\ t.q.\ \forall n>p\geq N\ ||f_n-f_p||^2=\sum\limits_{j\geq 0}|f_n(j)-f_p(j)|^2\leq \eps^2$\\
$\Rightarrow \quad |f_n(j)-f_p(j)|\leq \eps\ \forall j\in \N$.\\
$\Rightarrow \forall j\in \N\ (f_n(j)))_{n\in \N}$ est de Cauchy dans $\C$ qui est complet, donc $\exists f(j)\in \C$ telle que $\lim\limits_{n\to \infty} |f_n(j)-f(j)|=0$.

Il faut montrer que $f$ est la limite dans $l^2(\N)$ de la suite $f_n$.\\
$\forall \eps>0\ \exists N$ t.q. $\forall n>p\geq N \sum\limits_{j\geq 0}|f_n(j)-f_p(j)|^2\leq \eps^2$\\
$\Rightarrow$ $\forall J\in\N\ \underbrace{\sum\limits_{j=0}^J |f_n(j)-f_p(j)|^2}_{\text{somme partielle}}\leq \eps^2$, par passage à la limite sur $p$: $\sum_{j=0}^J|f_n(j)-f(j)|^2\leq \eps^2$

Conclusion: $\forall \eps>0\ \exists N$ telle que $\forall n\geq N\ ||f_n-f||<\eps \Longrightarrow \lim\limits_{n\to ∞}f_n=f$.	

Mais $f\overset{\text?}\in l^2(\N)$.

Vérifions que $f\in l^2(\N)$:\\$( \sum_{j\geq 0}|f(j)|^2 )^{1/2}=(\sum_{j\geq 0} |f_n(j)-f(j)+f(j)|^2)^\frac 12=||\underbrace{f-f_n}_{\in l^2(\N) }+\underbrace{f_n}_{\in l^2(\N)}||\leq ||f-f_n||+||f_n||<+∞$.
\end{example}



\begin{theorem}[Projection orthogonale]
	Soit $H$ un espace de Hilbert et $C$ une partie \texttt{convexe} \texttt{fermée} et \texttt{non vide} de $H$. Alors $\forall x\in H\ \exists ! y_0\in C$ t.q.
	\begin{enumerate}
		\item $\dist(x, C):=\inf\{d(x,y), y\in C\}=\inf\{||x-y||_H, y\in C\} = ||x-y_0||_H$
		\item $\forall y\in C\ \Re(x-y_0| y-y_0)\leq 0$ \question{why in the world would scalar product have values other than real}
	\end{enumerate} 
	$y_0$ est la projection orthogonale de $x$ sur $C$.
	
\end{theorem}


\begin{remark}
	\leavevmode
	\begin{enumerate}
		\item $C$ est convexe si $\forall x, y \in C$ $[x,y]=\{tx+(1-t)y, t\in[0,1]\}\in C$
		% img convexity
		\item $H=\R^2:\ [x,y]\in C$
		\item si $x_0\in C$ dans le cas $y_0=x_0$ et $\dist(x_0, C)=0=||x_0-x_0||_H$
	\end{enumerate}
\end{remark}





\begin{proof}
	Notons par $d=d(x,C)>0\ (x\in H\diagdown C)$. Soit $y, z\in C$ on pose $b=x-\frac12(y+z),\ c=\frac12(y-z):\ ||b||=||x-\frac12\underbrace{(y+z)}_{\in C}||\geq d$. On a aussi $b-c=x-y$ et $b+c=x-z$ $\Rightarrow ||x-y||^2+||x-z||^2=||b-c||^2+||b+c||^2=(b-c| b-c)+(b+c|b+c)=||b||^2+||c||^2-(b|c)-(c|b)+||b||^2+||c||^2 + (b|c)+(c|b)$.
	
	$||x-y||^2+||x-z||^2=2(||b||^2+||c||^2)\geq 2 d^2+2\frac14||y-z||^2 \Rightarrow ||y-z||^2\leq 2(||x-y||^2-d^2)+2(||x-z||^2-d^2)$. Pour $n\in N$ $C_n=\{y\in C ||x-y||^2\leq d^2+\frac1n\}$ est fermée dans H (boule fermée).
	
	Puisque $C$ est fermé, $C_n=\{y\in H ||x-y||^2\leq d^2+\frac1n\}\cap C$ est fermé dans $C$.
	De plus: $\delta (n):=sup\{||y-z||, (y,z)\in C_n\times C_n\}\leq * sup\{[2(||x-y||^2-d^2)+2(||x-z||^2-d^2)]^\frac12, y,z\in C_n$ $\Rightarrow$ $\delta (n)\leq \frac2{n^\frac12}\to 0$ quand $n\to +\infty$.

$H$ est complet et $C\subset H_x$ $c$ est fermé. $C$ est un espace métrique complet. Il satisfait le critère de Cantor: $\bigcap\limits_n C_n=\{y_0\}$.


$y_0\in\cup_n C_n\ d^2\leq ||x-y_0||^2\leq d^2+\frac1n\ \forall n\in\N*=\N\\\{0\}$
$\Rightarrow ||x-y_0||=d^2$.

Montrons ii): $\forall t\in[0,1],\ \forall\in H\ \phi(t)=||\underbrace{y_0+t(y-y_0)}_{\in C}-x||^2 = ||y_0-x||^2+2tRe(y_0-x|y-y_0)+t^2||y-y_0||^2$. $\phi(0)=d^2\leq \phi(t)\ \forall t\in(0,1]$ $\Rightarrow \phi'(0)\geq 0$. $\phi'(t)=2Re(y_0-x|y-y_0)+2t||y-y_0||^2$. $\phi'(0)\leq 0 \Rightarrow 2Re(y_0-x|y-y_0)\leq 0\Rightarrow (i)$.

\end{proof}


\begin{theorem}[corollaire]
	Soit $F$ un sous-espace \texttt{fermé} de $H$ alors: $H=F\oplus F^\perp$.
\end{theorem}
\begin{proof}
	\begin{itemize}
		\item $F$ est convexe puisque $\forall \alpha,\beta\in\C \forall x, y\in F\ \alpha x+\beta y\in F$ $\Rightarrow$ Cela est vrai si $\alpha = t,\ \beta=1-t\ t\in[0,1]$.
	
	On peut ceci appliquer le Thm 1:
		\item On a toujours $F+F^\perp \subset H$ et $F+F^\perp = F\bigoplus F^\perp$ car si $x\in F\cap F^\perp$ $\Rightarrow$ $(x|x)=0=||x||^2$ $\Rightarrow$ $x=0_H$
		
		Soit $x\in H$, et $y_0\in F$ sa projection orthogonale: $\forall d\in \C, y\in F, y_0+dy\in F$ et donc $Re(x-y_0| y_0+dy-y_0)\leq 0$ $\Rightarrow$ $Re(x-y_0|dy)\leq 0$
		
		$d=(x-y_0|y)$ $\Rightarrow$ $(x-y_0)$
		...
		
	Conclusion $Re(x-y_0|dy)$.. donc $H=F\bigoplus F^\perp$.
	\end{itemize}
\end{proof}

\begin{definition}
	Dans ces conditions, l'application $P:x\in H, x=x_1+x_2, x_1\in F, x_2\in F^\perp $ $$x \overset{P}{\mapsto} x_1\in F$$ est le \textsc{Projecteur Orthogonal} sur $F$.
\end{definition}

\begin{examplebox}
	Montrer que P est linéaire continue et satisfait $P^2=P$.
\end{examplebox}

\begin{definition}
	Une partie $A$ de $H$ est dite \textsc{Totale} si le plus petit sous espace fermé contenant $A$ et $H$.

	$H$ est \textsc{Séparable} si $H$ admet une famille totale dénombrable.
\end{definition}

\begin{examplebox}
	$H=l^2(\N): \mathcal{F}=\{e_0, e_1, ...\}$ avec $e_j(i)=\delta_{ij}\to (0,0,..., 0,1,0,... 0)$. $\mathcal{F}$ est totale. Elle est dénombrable, $l^2(\N)$ est séparable.
\end{examplebox}

\begin{theorem}
	Soit $H$ un espace de Hilbert et $A\subset H$:
	\begin{enumerate}
		\item $\overline{\vect(A)}=(A^\perp)^\perp$
		\item $A$ est un sous-espace alors $(A^\perp)^\perp=\bar A$
		\item $A$ est totale $\Leftrightarrow$ $A^\perp=\{0_H\}$
	\end{enumerate}
\end{theorem}

\section{Séries dans un espace vectoriel normé} % (fold)

Soit $(E, ||\cdot||_E)$ un \texttt{espace vectoriel normé} (e.v.n).
\begin{definition}
	On appelle \textsc{Série} de terme général $u_n\in E$ la suite $(S_N)_{N\in \N}$ de $E$ t.q.  $S_N=\sum\limits_{n=0}^Nu_n$.
	La série est \textsc{Convergente} dans $(E, ||\cdot||_E)$ si la suite $(S_N)_{N\in\N}$ admet une limite dans $E$: $S$ --- c'est la somme de la série.
\end{definition}



\begin{definition}
	Une série $\sum u_n$ est dite \textsc{Absolument Convergente} (AC) si la série $\sum ||u_n||_E$ est convergente dans $\R^+$.
\end{definition}

\begin{theorem}
	Si $E$ est \texttt{complet} (espace de Banach/Hilbert) Alors toute série AC est convergente et $||\sum\limits_{n=0}^\infty||\leq \sum\limits_{n=0}^\infty||u_n||$.
\end{theorem}
\begin{proof}
	$J_n=\sum\limits_{n=0}^N||u_n||$ et convergente $\Leftrightarrow$ $(J_n)_{N\in\N}$ est de Cauchy $\forall \eps >0\ \exists K\ t.q.\ \forall N>P\geq K\Rightarrow |J_n-J_p|\leq \eps$. $\sum_{j=p+1}^N||u_j||\leq \eps$. meus $||S_n-S_p||=||\sum_{j=p+1}^Nu_j||\leq\sum_{j=p+1}^N ||u_j||$ inégalité triangulaire.
	
	$\Rightarrow N>p\leq K\Rightarrow ||S_N-S_P||\leq \eps \Leftrightarrow (S_N)_{N\in \N}$ est de Cauchy dans $E$ et donc convergente.
	
	D'autre part $||S_n||=||\sum_{j=0}^n u_j||\leq\sum_{j=0}^n\leq \sum_{j=0}||u_j||$ $\Rightarrow\ ||\sum_{j=0}u_j||\leq \sum_{j=0}||u_j||$. Cfd.
\end{proof}

\begin{definition}

	Une suite $(x_n)_{n\in\N}{n\in \N}$ de H est dite \textsc{Orthogonal} si $(x_i|x_j)=0\ \forall i≠j$.
\end{definition}

\begin{theorem}
	Soit $(a_n)_{n\in \Z}$ une suite orthogonale dans un espace de Hilbert $H$. Alors la série $∑x_n$ est convergente $\Longleftrightarrow$ $∑_{n≥0}||x_n||_H^2$ est convergente et \[ ||∑_{n≥0}x_n||^2_H=∑_{n≥0}||x_n||_H^2.\]
\end{theorem}
\begin{proof}
	$\forall$ $l>p$ on a $||∑_{n=l}^p||^2=(∑_n=e^p x_n | ∑_n=e^p x_n)=∑_n,n'=l(x_n|x_n')=∑_n=l^p||x_n||^2$ Alors $(x_n)_{n\in\N}{n\in \N}$ est de Cauchy $\Leftrightarrow$ $(||x_n||^2)_{n\in \N}$ est de Couchy dans $\R$.

D'autre part $S_N=∑_{n≥0}^N x_n$ $\Rightarrow$  $||S_N||^2=∑_{n≥0}^N||x_n||^2$.
Alors $S=lim S_N=∑x_n$ $||S||^2=||lim N S_N||^2=lim ||S_N||^2$ par continuité de la $||•||$ et donc $||S||^2=lim_N∑_n≥0^N||x_n||^2=∑_{n≥0}||x_n||^2$
\end{proof}

\section{Bases Hilbertiennes} % section #3

\begin{definition}

	On appelle \textsc{Base Hilbertienne}, une suite de vecteur $(x_n)_{n\in\N}{n\in \N}$ telle que 

	\begin{enumerate}
		\item $\forall n, m (x_n|x_m)=δ_{nm}$,
		\item $\vect\{(x_n)_{n\in\N}n\in\N\}=H$ $\Leftrightarrow$ $\vect{(x_n)_{n\in\N}{n\in \N}}^\perp=\{0_H\}$ $\Leftrightarrow$ $(x_n)_{n\in\N}{n\in \N}$ est totale.
	\end{enumerate}
\end{definition}

\begin{theorem}[Inégalité de Bessel]
	Soit $(x_n)$ une suite \texttt{orthonormale} ($\forall n, m (x_n|x_m)=δ_{nm}$) dans $H$. Alors $\forall x\in H ∑_{n≥0}|(x|x_n)|^2$ est convergente et $∑_{n≥0}|(x|x_n)|^2≤||x||^2$.
\end{theorem}

\textbf{Exemple:} $H=l^2(\N)$. $(e_n|e_m)=∑_{k≥0}e_n(k)\overline{e_m(k)}=∑_{k≥0}δ_{nk}δ_{mk}=δ_{nm}$. En fait on montre que $∑_{n≥0}|(e_n|x)|^2=||x||^2$ c'est une base Hilbertienne.
\begin{proof}
	Soit $x\in H$ on pose $y_i=(x|e_i)e_i$ et $Y_N=∑_1^Ny_i$, $Z_N=X-Y_N$. Alors: $(Z_N|y_i)=(X-Y_N|y_i)=(X|y_i)-(Y_N|y_i)$. $(x|y)=(x|(x|e_i)e_i)=\overline{(x|e_i)}(x|e_i)=|(x|e_j)|^2$. $(Y_N|y_i)=∑_{j=1}^N(y_j|y_i)$ mais $y_j\perp y_i$ $\Rightarrow$  $(Y_N|y_i) =||y_i||^2$ si $N≥i$.
	(autrement =0)

Dans ces conditions puisque $||y_i||^2=|(x|e_i)|^2$. Alors $(Z_n|y_i)=0$ $\Rightarrow$  $(Z_N|Y_N)=0$ cas $Y_n=∑_{i=0}^Ny_i$ $\Rightarrow$  $||x||^2=||Z_n||^2+||Z_N||^2$ $(x=Zn+Yn et Z_n\perp Y_n)$
$\Rightarrow$  $||y_n||^2=∑||y_n||^2≤||x||^2$


La seuie $∑^N||y_n||^2$ est positive, majorée donc convergente et par passage à la limite: $∑_{n≥0}||y_n||^2=∑|(x|e_n)|^2≤||x||^2$. QED


\end{proof}


\begin{theorem}[Egalité de Parseval]
	Soit $(e_n)$ une base Hilbertienne de $H$ alors 
	\begin{enumerate}
		\item La série $∑_{n≥0}|(x|e_n)|^2$ est convergente et $||X||^2=∑_{n≥0}|(x|e_n)|^2$,
		\item La série $∑_{n≥0}(x|e_i)e_i$ est convergente dans $H$ et $∑_{i≥0}(x|e_i)e_i=x$.
	\end{enumerate}
\end{theorem}
\begin{proof}
	En utilisant le théorème précédent alors $∑|(x|e_i)|^2$ est convergent. On utilise l'identité de la médiane: $∑(x|e_i)e_i$ est convergente dans $H$ $(||(x|e_i)e_i||^2 =|(x|e_i)|^2)$.
	On pose $y=∑_{i≥0}(x|e_i)e_i$ alors $||y||^2 =∑_{i≥0}|(x|e_i)|^2)$ mais $(y|e_j)=(∑(x|ei)ei|e_j)=∑(x|e_i)(e_i|e_j)=(x|e_j)$ ...
	Conclusion $\forall j\in \N (x|e_j)=(y|e_j)$ $\Leftrightarrow$
	$ (x-y|e_j)=0$ $\Rightarrow$  $x-y\in\vect((e_n)_{n\in\N})^\perp$
	$\Rightarrow$  $x-y=0_H$ $\Leftrightarrow$ $x=y=∑(x|e_i) e_i ||x||^2=∑_{i≥0}|(x|e_i)|^2$
\end{proof}
\begin{remark}
	Si $(e_n)_{n\in \N}$ est une suite orthonormale telle que $\forall x\in H x=∑_{i≥0}(x|e_i)e_i:\ x=\lim_N ∑_{i≥0}^N a_ie_i$ où $a_i=(x|e_i)\in\C$ 
	
	
	$\in \vect\{(e_n)_n\in \N\}; a_i=(x|e_i)$ $\Rightarrow$  $\vect\{(e_n)_n\in\N\}=H$. $(e_n)_n\in \N$ est une base Hilbertienne.
	ii)>> $(e_n)_n\in\N$ est base Hilbertienne de $H$ $\Leftrightarrow$ $\forall x\in H:\ ∑(x|e_i)e_i=x $
	$∑(x|e_i)e_i=x$ $\Leftrightarrow$ $∑|(x|e_i)|^2=||x||^2 $i $>>$ $(e_n)$ est une base Hilbertienne de $H$ $\Leftrightarrow$ $∑|(x|e_i)|^2=||x||^2 \forall x\in H$
\end{remark}


Exemple (suite):
$H=l^2(\N)$. $(e_n)_{n\in \N} t.q. e_n(k) =δ_{nk}$.

$u\in H$ $\Leftrightarrow$ $∑_{n≥0} |u(n)|^2=||u||^2$ mais $u(n)=(u|e_n)=∑u(k)e_n(k)$ $\Leftrightarrow$ $∑_n≥0 |(u|e_n)|^2=||u||^2$, $\Rightarrow$  c'est une base Hilbertienne. !?

\section{Dual d'un espace de Hilbert} % (fold)

On rappelle que si $S$ est un e.v.n. une \textsc{Forme Linéaire} sur $X$ est une application linéaire de $X$ dans $\C$. Soit $l: X \rightarrow  \C:\ \forall d \in \C\ \forall x, y\in X l(x+dy)=l(x)+dl(y)$. L'ensemble des formes linéaires de $X$: est un espace vectoriel $X^*$. On considère $X'$ dual topologique: c'est l'espace vectoriel des formes linéaires continues sur $X:\ \{l:(X,||•||_X)\rightarrow (\C, |•|)\}$.

\begin{exercise}	
	$l$ est continue $\Leftrightarrow$ 
	\[\exists C>0\ x \forall x\in X, |l(x)|≤C||x||\label{eq:cont} \tag{\textasteriskcentered}\]
\end{exercise}

On définit $l\in X'$, $||l||=\inf\{C>0 \text{ t.q. \eqref{eq:cont} est satisfait}\} = \sup\{ |l(x)|\ |\ ||x||=1\}$.
$(X', ||•||)$ est un espace de Banach (un e.v.n. \texttt{complet})



\begin{theorem}[Théorème de représentation de Riez]. Soit $H$ un espace de Hilbert $H'$ son dual topologique. On définit $I :H\rightarrow H"$ par $\forall x\in H I(x)=(•|x)$. Alors $I$ est un isomorphisme isométrique de $H\rightarrow H'$.
\end{theorem}

\begin{remark}
	$H=\C^n$, une forme linéaire sur $\C^n$: $l$. 
	$l(x_1,...\, ,x_n)=∑_{i=1}^n a_ix_i,\ a_i\in \C$
	$|l(x)|=|∑_{i=1}^na_ix_i|≤sup\{a_i|\}•||x||_{\R^n}$. Ici $X^*=X'$ !?

	$$l(x)=(a_1,a_2,...\,, a_n)\mqty({x_1\\ x_2\\ \vdots\\ x_n})$$
	$=(\bar a|x) \forall x\in \C^n$
	$\forall l\in X',\ \exists a\in \C:\quad l(x)=(x|\bar a)$
	Généralisation à la dimension quelconque c'est le théorème de Riez:
	$\forall l\in H'\ \exists a\in H\  \forall x\in H:\ l(x)=(x|a)|$
\end{remark}
 

\begin{proof}
	
	
	Soit $l\in H'$ $l≠0_h'$ $\Leftrightarrow$ 
	
\ifcomment

	lei $l≠H$ pueque $\exists \in tq l(x)≠0_h$ On Satit que $\ker l$ est ferme sait $(x_n)_{n\in\N}n\in|N$ une suiti de kei $l$ convergente dans $H$: $x_n\rightarrow x\in H$ mais latren time: $l(x_n)\rightarrow  l(l(x)$ mais l(x_n)=0 \forall n. $\Rightarrow$  l(x)=0 x\in ker Alors H=ker l \oplus (ker l )^perp (thm propilere
	puisque (ker l)≠H $\Rightarrow$  \ker lY\perp–\{0--_h]) Soit x\in ker \phi ^\perp, ||x||=1|)| x≠0_H

	
	\forall y\in H soit z=-l(x)y_l)y)x\in H et l(lx)=-l(x_l(y)+l(y)l(x)=0 x\in rerl $\Rightarrow$  (x|z)=0
	
	$\Rightarrow$  )x|-l(x)y+l(y)x) $\Rightarrow$  l(x) \rightarrow  l(x)(y|x)=l(y)(X|X) $\Rightarrow$  \forall y\in H l(y)=(y|\overline{l(x)X)|)|))
	

	$\forall l\in H' \exists a\in H$ t.q. $\forall x\in H l(x)=(x|a)|$ I est surjective. Montres que I est injective.
	Soit $x\in H$ t.q. $I(x)=O_H'$ $\Leftrightarrow$ $\forall y\in H I(x)(y)=(y|x)=0$ $\Rightarrow$  $x\perp H$ $\Rightarrow$  $X=0_H$ $\ker I=\{O_h\}$ $I$ est injective donc bijection.




Enfin: $||I(x)||=\sup\P|(y|x)|, ||y||=1\} -||x|| isométrie.)|$

Parce que $|(y|x)|≤||y||$ $||x||=||x||$ $y=\frac x{||x||}$ $||y||=1$ $|(y|x)|=||x||$
\fi

\end{proof}
\begin{remark}
	Si l est anti-linéaire: $\forall d\in \C\ \forall x,y\in H\ l(x+dy)=l(x)+\bar d l(y)$ et $\exists u $ t.q. $\forall x\in H:\  l(x)=(u|x)$
\end{remark}


\section{Convergence faible dans les espaces de Hilbert} % (fold)

\subsection{Définition et premières propriétés} % (fold)
\label{sub:definition_et_premieres_proprietes}

\begin{definition}
	Soit $H$ un espace de Hilbert. Une suit$ (x_n)_{n\in\N}$ de $H$ est dit \textsc{Converge Faiblement vers} $x\in H$ si $\forall y\in H (x_n|y)\rightarrow (x|y)$. On notera $x_n\rightharpoonup x$, $x$ est dite limite faible de $(x_n)_{n\in\N}$.
\end{definition}
Exp. $H=l^2(\N)$, $x_n\in l^2(\N^*)$ t.q. $x_n(j)=δ_{nj}$.

$(x_n)_{n\in\N}n\in\N$ est une base hilbertienne de H. On regarde la convergence faible. Soit $y\in l^2(\N^*)$ on doit calculer $\lim_{n\to +∞}(x_n|y)$, $(x_n|y)=∑_j x_n(j)\overline{y(j)}=\overline{y(n)}$. $|(x_n|y)|≤|y(n)|$ on sait $∑_j|y(j)|^2<+∞$ $\Rightarrow$ $|y(j)|\to 0$ qd $j\to+∞$ et donc $|(x_n|y)|=|y(n)|\to 0$ qd $n\to +∞$. On ercit $0=(0_H|y)$ alors $\lim_n(x_n|y)=(0_H|y)$. $0_H$ est une limite de la suite $(x_n)_{n\in\N}{n\in \N}$ (On montrera la limite faible est unique).
$\norm{x_n}^2=∑_j|x_n(j)|^2=1$ $\Rightarrow$ $x_n\not\to 0$ puisque $\lim_n\norm{x_n-0_H}=\lim_n\norm{x_n}=1\not\to 0$. $0_H$ n'est pas limite de la suite $(x_n)_{n\in\N}$.

\begin{proposition}
	La limite faible, si elle existe elle est unique.
\end{proposition}
\begin{proof}
	Supposons que $\forall y\in H (x_n|y)\to (x|y)$ et $(x_n|y)\to (x'|y),\ x,x'\in H$. Supposons $x\neq x'$ $\Leftrightarrow$ $x-x'≠0_H$ $\Rightarrow$ $\exists y\in H$ t.q. $(x|y)≠(x'|y)$ (*)
		\begin{remark}
			On suppose (*) faux: $\forall y\in H (x|y)=(x'|y)$ $\Leftrightarrow$ $(x-x'|y)=0$ $\Rightarrow$ $x-x'\perp H$ $\Rightarrow$ $x-x'=0_H$ c'est Absurde.
		\end{remark}
	On pose $u_n=(x_n|y)$ $u=(x|y)$ $u'=(x'|y)$
	$u_n\to u:\ \forall  ε>0\ \exists N$ t.q. $\forall n≥N |u_n-u|≤ε$. On choisit $ε<|u-u'|$ alors on a toujours si $n≥N$ $|u_n-u'|=|u_n-u+u-u'|=||u-u'|-|u_n-u|| ≥|u-u'|-ε≥\frac{|u-u'|}2$ $\Rightarrow$ $\forall n≥N |u_n-u'|≥\frac{|u-u'|}{2}$ $\Rightarrow$ $|u_n-u'|\not\to 0$ $\Leftrightarrow$ $u_n\not\to u'$ QED.
\end{proof}
Dans l'exemple précédent $0_H$ est la limite unique de la suite $(x_n)_{n\in\N}$

Exemple. $H=L^2(\R)$. Soit $H_0\in C^∞_c(\R)$ On pose $\forall n\in \N$, $φ_n(x)=φ_0(x-n)\ x\in\R$.

\begin{rappel}
	
	$C_c^∞(\R)$ ensemble des fonctions $f:\R\mapsto  \C$. \\
	* support $f$ compact : borne et ferme.\\
	* $f\in C^n_(\R)$
	$\Leftrightarrow$ $f\in C_X^∞(\R)$
	support $f=\overline{\{x\in \R, f(x)≠0\}}$
	
	$L^2(\R)=\overline{C_x^∞(\R)}|_{\norm{•}_{L^2(\R)}}$
\end{rappel}

$φ_0\in C_C^∞(\R)$, $\forall n\in\N φ_n(x)=φ_0(x-n)$.

$\forall \psi\in L^2(\R)$: $(φ_n|\psi) \to 0=(0_H|\psi)$ $(φ_n|ψ) = ∫_\R \dd{x} φ_n(x) \overline{ψ(x)} = ∫_{n-1}^{n+1} \dd{x} φ_0(x-n)\bar ψ(x)$.  $|(.|.)|_{L^2((n-1,n+1))}≤\norm{•}\norm{•}$ $\Rightarrow$  $∫_{n-1}^{n+1} |φ_0(x-n)|^2 \dd{x} = ∫_{-1}^{+1} |φ_0(t)|^2 \dd{t} =1$ $\Rightarrow$ $|(φ_n|ψ)|≤(∫_{n-1}^{n+1}|ψ(x)|^2\dd{x})^{\frac 12}$

$ψ\in L^2(\R)$ $\Rightarrow$ $∫_{n-1}^{n+1} |ψ(t)|^2 \dd{t} \to 0$ quand $n\to +∞$. $\norm{ψ}=∑_n∫_{n-1}^{n+1}$ $|ψ|^2\dd{t}<∞$.

\begin{proposition}
	\begin{enumerate}
		\item soit $(x_n)_{n\in\N}$ t.q. $x_n\rightharpoonup x \in H $alors $(x_{k(n))})_{n\in\N}$ Converge faiblement et $x_{k(n)}\rightharpoonup x$
		\item si $(x_n)_n\in\N$ et $(y_n)_{n\in\N}$ sait deux suites t.q. $x_n\rightharpoonup x$ et $y_n \rightharpoonup y$ alors $x_n+y_n\rightharpoonup x+y$
		\item si $x_n\rightharpoonup x$ et soit $(d_n)_{n\in\N}$ une suite des $\C$ t.q. $d_n\to d \in \C$ $\Rightarrow$ $d_nx_n\rightharpoonup dx$.
	\end{enumerate}
\end{proposition}
\begin{proof}
	\begin{enumerate}
		\item i est évident $\forall y\in H$ si $u_n=(y|x_n)$ $\Rightarrow$ $u=(y|x)$ $\Rightarrow$ $u_{k(n)}\to u$ $\Rightarrow$ i)
		\item $\forall y\in H (y|x_n+z_n)=(y|x_n)+ (y|x_n) \to (y|x)+(y|z)=(y|x+z)$.
		\item On suppose $\forall y\in H (x_n|y)\to (x|y)$ et $d_n\to d$.
		$(d_nx_n-dx|y)=(d_nx_n-dx_n+dx_n-dx|y)=(d_n-d)(x_n|y)+d(x_n-x|y)$ $\Rightarrow$ $|(d_nx_n-dx|y)|≤|d_n-d||(x_n|y)|+|d||(x_n-x|y)|$
		\begin{enumerate}
			\item $(x_n|y)\to (x|y)$ $\Rightarrow$ $\exists M$ t.q. $|(x_n|y)|≤M\ \forall n\in\N$ $\Rightarrow$ $|d_n-d||(x_n|x)|≤|d_n-d|M\to 0 qd n\to +∞$. 
			$|(x_n-x|y)|\to 0 qd n\to +∞$ par (*) la proposition est démontrer.
		\end{enumerate}
	\end{enumerate}
\end{proof}
\begin{remark}
	On a toujours que $|(x_n-x|y)|≤\norm{x_n-x}_H\norm{y}_H$. Si $\lim_n\norm{x_n-x}=0$ $\Leftrightarrow$ $\lim_n x_n=x$ $\Rightarrow$ $x_n\rightharpoonup x$
	! l'inverse est faux en général.
\end{remark}
\begin{proposition}
	Si $x_n\rightharpoonup x$ dans $H$ alors $\lim_{n\to + ∞}\inf\norm{x_n}≥\norm{x}$.
\end{proposition}
\begin{remark}
	Si $(x_n)_{n\in\N}$ converge $\exists x\in H$ et $\lim_{n\to +∞}\norm{x_n-x}=0$ alors par $|\norm{x}-\norm{x_n}|≤\norm{x-x_n}$ $\Rightarrow$ $\lim_{n\to ∞}\norm{x_n}=\norm{x}$.
	Mais si on a que $x_n\rightharpoonup x$ on ne sait pas que la suite $\norm{x_n}$ converge, c.a.d. que la limite existe par contre $\lim_n\inf\norm{x_n}$ = $\lim_{n\to ∞}\inf\{\norm{x_k}, k≥n\}$ et $\lim_n\sup\norm{x_n}-\lim_{n\to +∞} \sup\{\norm{x_k}, k≥n\}$ existe toujours.
\end{remark}
\begin{proof}
	Puisque $x_n\rightharpoonup x$, alors $(x_n|x)\to (x|x)=\norm{x}^2$ en utilisant Cauchy Schwartz $|(x_n|x)|≤\norm{x_n}{x}$.
	$\Rightarrow$ $\norm{x}^2≤\norm{x_n}\norm{x}$ $\Leftrightarrow$ $\norm{x}≤\norm{x_n}$ $\Rightarrow$ $\norm{x}≤\lim_{n\to∞}\inf\norm{x_n}$.
\end{proof}
\begin{proposition}
	Soit $(x_n)_{n\in\N}$ une suite dans $H$. Alors 
		$x_n\to x$ $\Leftrightarrow$ $x_n\rightharpoonup x$ et $\lim_n\sup\norm{x_n}≤\norm{x}$
\end{proposition}
\begin{proof}
	($\Rightarrow$) $x_n\to x$ $\Rightarrow$ $x_n\rightharpoonup x_n$ et $\norm{x_n}\to \norm{x}$
	($\Leftarrow$) $\norm{x-x_n}^2 = \norm{x}^2+\norm{x_n}^2 - 2\Re (x|x_n)$
	$\lim_n\sup \norm{x-x_n}^2≤ \norm{x}^2+\lim_n\sup\norm{x_n}^2 - 2\norm{x}^2$.
	$\lim_n\sup \norm{x-x_n}^2≤\lim_n\sup \norm{x_n}^2-\norm{x}^2 ≤0$
	$\Rightarrow$ $\lim_n\sup \norm{x-x_n}^2=0 ≥\lim_n\inf \norm{x-x_n}^2≥0$
	$\Rightarrow$ $\lim_n\sup \norm{x-x_n}^2 =\lim_n\inf\norm{x-x_n}^2=\lim_n\norm{x}$
\end{proof}

\begin{example}
	Soit $(x_n)_{n\in\N}$ une suite bornée de H. Soit $D\subset H$ dense ($\bar D=H$). Alors $x_n\rightharpoonup x$ sur $H$ $\Leftrightarrow$ $(x_n|y)\to (x|y)\ \forall y\in D$.
\end{example}

\begin{exercise}
	On considère $H=L^2(\R, dx)$, soit $φ\in H $$\Leftrightarrow$ $∫_\R|φ|^2\dd{x}=\norm{φ}^2_{L^2(\R)}$. $H=\overline{C_c^∞(\R)}$
\end{exercise}
Soit $φ_0\in C_0^∞(\R)$ tq $\norm{φ_0}_{L^2(\R)}=1$ (sinon on pose $φ=\frac{φ_0}{\norm{φ_0}}, \norm{φ}=1$)
On pose $φ_n(x)=φ_0(x-n)$, on veut montrer que $φ_n\rightharpoonup φ\in L^2(\R)$
On remarque que:
$\norm{φ_n}^2=∫_\R|φ_0(x-n)|^2\dd{x}$
On pose $u=x-n$:
$\norm{φ_n}^2=∫_\R\dd{u}|φ_0(u)|^2=1$
$φ_n\not\to0 \norm{f_n-0}=1$.

Est ce que la suite conv faiblement?
$\exists φ\in H, (φ_n|ψ)\rightarrow (φ|ψ)\forall ψ\in H.$

Soit $ψ$: $ψ(x)=1$ ssi $x\in[-1,1]$ $ψ(x)=0$ sinon.
$∫_\R|ψ(x)|^2\dd{x}=∫_{-1}^11\dd{x}=2$
On choisit $n≥N$ avec $N$ tq $a+N≥$
$\Rightarrow$ $∫_\Rφ_nψ\dd{x}=0$
On a montré $(f_c|ψ)\rightarrow 0=(0|ψ)$. Question $φ_n\rightharpoonup 0_{L^2(\R)}$?

\begin{proposition}
	Soit $H$ un espace de Hilbert $D\subset H$ dense dans $H$: $\bar D=H$. Alors soit $(x_n)_{n\in\N}$ une suite borné dans $H$, $x_n\rightharpoonup x\in H$ $\Leftrightarrow$ $(x_n|y)\to(x|y)$ $\forall y\in D$.
\end{proposition}
Exercice(suite) On doit monter que $\forall ψ\in C^2(\R)$: $(φ_n|ψ)\to0$. On remarque que $\norm{φ_n}=1$ $\forall n\in\N $donc elle est bornée.
(Suite bornée: $\exists Π tq\forall n \norm{x_n}≤Π$)
Il suffit de montrer $(φ_n|φ)\to 0 \forall ψ\in C_0^∞(\R)$.
Montrons a dernier point:
$∫_\Rψ(x)φ_n(x)\dd{x}$; $\exists a,b \in \R, supp ψ\subset[A,B]$.
On chοisit $Ν$ tq supp $φ_N=[a+N,b+N]$, $a+Ν>Β$
$\Rightarrow$ $∫_\Rψφ_n=0$ $\Rightarrow$ $\lim_n(ψ|φ_n)=0=(ψ|0)$
\begin{proof}
	Si $φ_n\rightharpoonup φ$ dans $H$ $\Rightarrow$ $φ_n\rightharpoonup φ$ dans $D$. Supposons que $(φ_n|ψ)\to (φ|ψ)\forall ψ\in D$.
	Soit $η\in H \exists(η_k)_{k\in\N}$ suite de $D$ tq $\lim_n\norm{η_k-η}=0$.
	On calcul $(φ_n|η)=(φ_n|η_k)+(φ_n|η-η_k)$.
	Soit $ε >0$, $\exists K$ tq si $k>K \norm{η-η_k}≤\fracε2$
	alors $|(φ_n|η-η_k)|≤\norm{φ_ν}\norm{η-η_k}≤Πε$. On fixe un tel $k$.
	On conclut que $\forallε>0$, $\exists N$ tq si $n≥N$; $|(φ_n|η)|≤(Π+1)ε$ $\Rightarrow$ $(φ_n|η)\to 0$.
\end{proof}
\begin{theorem}[1]
	Toute suite faiblement convergente dans un espace de Hilbert est \texttt{bornée}.
\end{theorem}
\begin{theorem}[2, Banach-Alaoglu-Bourbaki]
	Une espace de Hilbert vérifie la propriété de Bolzano-Weierstrass faible. De toute suite bornée de $H$, en peut extraire une sous suite.
\end{theorem}
\begin{remark}
	Dans $\R$, de toute suite borné on peut extraire une sous-suite c.v. (B.W.) c'est vrai si $p<+∞$. Mais c'est faux en dimension quelconque. Le Thm 2 \~> c'est vrai au sous faible.
\end{remark}
\begin{proof}
	Soit $(x_n)n\in N$ une suite borné dans $H$: $\exists L>0$ tq $\forall n\in \norm{x_n}≤L$. Soit $M=\overline{\vect(x_n)}$. Si $M$ est de dimension fini, alors $(x_n)_{n\in \N}\subset B_f(0_M,L)\subset M$. qui est compact $\Leftrightarrow$ elle satisfait la propriété de B.W. $\exists (X_{k(n)})_{n\in\N}$ sous suite et $x\in B_f(0, L)$ tq $\lim_n\norm{x_{k(n)}-x}\to 0$ $\Rightarrow$ $x_{k(n)}\rightharpoonup x$ dans $H$. Alors le Theoreme 2 est démontré.
	Supponons que $M$ n'est pas de dimension finie.
	$M$ est un espace Hilbert (sous espace ferme de $H$) Soit $(φ_k)_{k\in\N}$ une base hilbertiere de $M$. La suite $((x_n|e-1))_{n\in\N}$ est bornee car $|(x_n|e_1)|≤\norm{x_n}\norm{e_1}≤L•1=L$
	On appleque la proprieté de B.W. dans $\C$: $\exists(a_{k(n)})_{n\in\N}$ et $c_1\in\C$ tq $a_{k(n)}\to c_1$ qd $n\to+∞$ on réécrit: $a_{k(n)}$ on pose $x_{k(n)}=x_n'$. $\forall n\in\N$ alors $(x_n^1|e_1)\to c_1$ qd $n\to+∞$.
	2 la suite $(x_n'|e_2)$ est borné, $\exists$ une sous suite $(x_n^2)_{n\in\N}$ et $c_2\in\C$ tq $(x_n^2|e_2)\to c_2$ qd $n\to+∞$ etc...
	
	Canclusion: On a construit des sous suité
	$(x_n)_{n\in\N}\subset(x^1_n)_{n\in\N}\subset...(x^k_n)_{n\in\N}...$
	et des complexes $C_k$, $k=1,2,3...$ tq $(x_n^k|e_k)\to c_k$ qd $n\to+∞$.
	(présidé deogonal de Cantor): on pose $z_n=x_n^n$.
	Montrer que $z_n\rightharpoonup ∑_kc_ke_k$ si $∑_kc_ke_k$ est conv dans $H$. Le thm 2 est démontré. Montrons que $∑_kc_ke_k=z\in M$ i.e (*).
	Puisque $M$ est complet alors il faut montrer $S_n=∑_{k=1}^nc_ke_k$ est de Cauchy: $\norm{s_n-s_m}^2=\norm{∑_{k=n+1}^mc_ke_k}^2=∑_{k=n+1}^m|c_k|^2$ (Parseval).
	$S_n$ est de Cauchy $\Leftrightarrow$ $\tilde S_n=∑_{k=1}^n|c_k|^2$ est de Cauchy $\Leftrightarrow$ $\tilde S_n$ est convergent dans $\C$.
	Montrons ce dernier point. On utilise l'inégalité de Bessel. 
	$∑_{k=1}^N|(x_n|e_k)|^2≤\norm{z_n}^2≤L^2$ mais: $(z_n|e_k)+(x_n^n|e_k)\to c_k$ qd $n\to+∞$. puisque $(x_n^n)_{n\in\N}$ est une sous suite de $(x_n^k)_{n\in\N}$ pour $n≥k$.
	
	$(x_n)_{n\in\N}\subset(x_n^2)_{n\in\N}\subset...(x_n^k)_{n\in\N}\subset(x_n^{k+1})_{n\in\N}...$
	$x_1^1$	$x_2^2$ 		... $x_k^k$
	alors $\lim_{n\to+∞}(x_n^n|e_k)=c_k$. Alors
	$∑_{k=1}^N|c_k|^2=∑_{k=1}^N\lim_n|(x_n^n|e_k)|^2 = \lim_n∑_{k=1}^N|(x_n^n|e_k)|^2=\lim_n ∑_{k=1}^N|(z_n|e_k)|^2$
	 on utilisant (*) alors $∑_{k=1}^N|c_k|^2≤L^2$ (par passage à la limite)
	 Par conséquent $∑|c_k|^2$ est convergente donc $∑_{k≥1}c_kφ_k$ est convergente dans $M$. Soit $z=∑_{k=1}c_kφ_k$ alors $(z|e_c)=c_e$. Alors on a montre que $\forall C\in \N^*$
	 $(z_n|e_c)\to c_e=(z|e_c)$
	 En utilisant que $\overline{vect(e_k, k\in \N^*}=M$ et $(x_n)_{n\in\N}$ est bornée alors cela entraine la convergence faible sur $M$.
	 $\forall y\in M: (x_n^n|y)\to(z|y)$
	 On a couverture une sous suite de $(x_n)_{n\in\N}$ qui conv faiblement sur $M$ vers $z\in H$. On étend la propriété sur $H$: $M$ est un sous espace fermé on lui applique le théorème des ces projection. $\forall η\in H M \exists!y_0\in M$ projection de $y$ sur $M$.
	 
	 Alors $y=y_0+(y-y_0)$ et $(x^n|y)=(x_n|y)=(x_n|y_0)+(z_n|y-y_0)$ mais $(z_n|y-y_0)=0$. $z_n\in M$ et $y-y_0\in Π^\perp$ $\Rightarrow$ $limit_n (z_n|y)=(z|y_0)$ ( ce que l'on a démontré précédent)
	 mais $z\in M$, donc $(z|y-y_0)=0$: $\lim_n(x_n|y)=(z|y_0)+(z|y-y_0)=(z|y)$ ce qui montre la conv faible sur $H$.
\end{proof}

\begin{theorem}[Completion]
    Si $(\mathcal{V}, (•|•)_\mathcal{V})$ est un espace préhilbertien, alors, il existe un espace de Hilbert $(\mathcal{H}, (•|•)_\mathcal{H})$ et une application $U:\mathcal{V}\rightarrow\mathcal{H}$ que:
    \begin{enumerate}
        \item $U$ est bijective
        \item $U$ est linéaire
        \item $(Ux|Uy)_H=(x|y)_\mathcal{V}\ \forall x\in \mathcal{V},\ \forall y\in \mathcal{V}$
        \item $U(\mathcal{V})=\{Ux\ |\ x\in\mathcal{V}\}$ est dense dans $\mathcal{H}$.
    \end{enumerate}
\end{theorem}

\begin{theorem}
	Soit $(E,(•|•))$ une espace préhilbertien. Soit $(v_n)_{n\in\N}$ une famille libre de $E$. Alors il existe une famille orthonormale de $E$, telle que:
	\begin{itemize}
		\item $\vect((e_n))=\vect((v_n))$
		\item $(e_n|v_n)>0$, $\forall n\in\N^*$
	\end{itemize}
\end{theorem}

\textbf{Procédé de Gram-Schmidt.}\\
Soit $u_1=v_1$, et $e_1=\frac{u_1}{\norm{u_1}}$; $u_2=v_2-\frac{(v_2|u_1)}{\norm{u_1}^2}u_1$, et $e_2=\frac{u_2}{\norm{u_2}}$; $u_3=v_3-\frac{(v_3|u_1)}{\norm{u_1}^2}u_1-\frac{(u_3|u_2)}{\norm{u_2}^2}u_2$ et $e_3=\frac{u_3}{\norm{u_3}}$ etc... 

\chapter{Opérateurs sur un espace de Hilbert} % (fold)
\label{cha:operateurs_sur_un_espace_de_hilbert}
\section{Généralités} % (fold)
\label{sec:generalites}
Soit $X,Y$ deux espaces de Banach, on note par $L(X,Y)$ l'ensemble des applications linéaires de $X\rightarrow Y$, si $X=Y$ on note par L(X).
Dans le cas d'espace de Hilbert l'ensemble des applications linéaires $L(\hs ,\hs')$ respectivement $L(\hs )$ si $\hs =\hs'$.\\
$T\in L(X,Y) N(T)=\{x\in X, Tx=0_y\}$ noll of $T$.\\
$R(T)=\{y\in Y,\exists x\in X et Tx=y\}$ range of $T$.\\
$G(T)=\{(x,Tx)\ x\in X\}$ graphe de $T$.
\begin{proposition}
	Soit $(X, \norm{•}_X)$ $(Y,\norm{•}_y)$ deux espaces du Banach soit $f\in L(X,Y)$, alors les assertions suivantes ont équivalentes.
	\begin{enumerate}
		\item $f$ est continue sur $X$
		\item $f$ est continue en un point $x_0\in X$
		\item $\exists C>0$ t.q. $\forall x\in X$ on a $\norm{Tx}_Y≤C\norm{x}_X$. 
	\end{enumerate}
\end{proposition}
\begin{proof}
	($\Rightarrow$) i)$\Rightarrow$ ii), montrons iii) $\Rightarrow$ i) on a $\forall x,y\in X\ \norm{f(x)-f(y)}_Y=\norm{f(x-y)}_Y≤C\norm{x-y}_X$ $\Rightarrow$ $f$ est Lipschitz sur $X$ donc continue.
	Montrons ii) $\Rightarrow$ iii) On choisit $x_0=0_X$ alors $f$ est continue en $0_X$. $\forall ε>0 \existsη(ε)$ t.q. $\forall x\in X$ et $\norm{x}_X≤η$ $\Rightarrow$ $\norm{f(x)-f(0)}_Y=\norm{f(x)}_Y≤ε$.
	Soit $ε=1$, soit $η=η(1)$, $\forall x\in X$ on pose $\tilde x=\frac η2\frac{1}{\norm{x}_X}$;
	On a $\norm{\tilde x}=\fracη2\frac1{\norm{x}_X}\norm{x}_X=\frac{η}2≤η$ $\Rightarrow$ $\norm{f(\tilde x)}≤1.$ Mais $f(\tilde x)=f(η\frac x{\norm{x}}=η\frac 1{\norm{x}}f(x)$
	et $\frac{η}{2\norm{x}_V}\norm{f(x)}_Y≤1$ $\Rightarrow$ $\norm{f(x)}_Y≤\frac 2η\norm{x}_X$ QED.
\end{proof}
\begin{remark}
	iii) $\exists C>0$ tq $\forall x\in X: \norm{f(x)}_Y≤C\norm{x}_X$ $\Leftrightarrow$ $\norm{f(\frac{x}{\norm{x}_X})}≤C$ si $\norm{x}_X≠0$ $\Leftrightarrow$ $x≠0_X$ $\Leftrightarrow$ $\forall x\in X \norm{x}_X=1$; $\norm{f(x)}_Y≤C$. $f(B_f(0_X, 1))\subset B_f(0_Y,C)$.\\
	* Un opérateur de $X\rightarrow Y$ et une application linéaire de $X\rightarrow Y$.\\
	* Une application linéaire $X\rightarrow Y$ continue est un opérateur borné de $X\rightarrow Y$.\\
	* On notera par $\mathcal{B}(X,Y)$ l'ensemble des opérateurs bornés de $X\mapsto Y$.
	
	Exp. $\hs=\hs'=l^2(\Z)$ on considère l'application $T:\hs\rightarrow \hs' \forall u\in \hs\ (Tu)(n)=u(n-1)$ shift a droite.
	$T$ est linéaire: $T(λu+μv)(n)=(λu+μv)(n-1)=λu(n-1)+μv(n-1)=λTu(n)+μTv(n)$
	$T$ est borné (donc continue) $\forall u\in \hs$
	$\norm{Tu}^2_\hs=(Tu|Tu)_\hs=∑_{n\in \Z}Tu(n)\overline{Tu(n)}=∑_{n\in\Z}u(n-1)\overline{u(n-1)}=∑_{l\in\Z}u(e)\overline{u(e)}=\norm{u}^2_\hs$ $\Rightarrow$ $\norm{Tu}_\hs=\norm{u}_\hs$ $T$ est une isométrie.
	
	$B(X,Y)$ est un espace normé, muni de la norme naturelle.
	$T\in B(X,Y)\ \norm{T}=\inf\{C>0 tq l'inégalité suivant est satisfait, \norm{Tx}≤C\norm{X}_X \forall x\in X\} \rightarrow  \norm{T}≥0$.
\end{remark}
\begin{exercise}
	Montrer que (*) définie une norme sur $B(X,Y)$.
\end{exercise}
\begin{proposition}
	Propriété Soit $T\in B(X,Y)$ alors
	\begin{align*}
	\norm{T}&=\sup\{\norm{Tx}_Y, \norm{x}_X=1\}\\
	&=\sup\{\norm{Tx}_Y, \norm{x}_X≤1\}\\
	&=\sup\{\norm{Tx}_Y, \norm{x}_X<1\}
	\end{align*}
	\begin{align*}
	\norm{T}&=\inf\{C>0 tq \norm{Tx}_Y≤C\norm{x}_X\}\\
	&=\inf\{C>0 tq \norm{T\frac x{\norm{x}_X}}_Y≤C \forall x\in X\}\\
	&=\inf\{C>0 tq \norm{Tx}_Y≤C \forall x:\ \norm{x}=1\}\\
	&=\sup\{\norm{Tx}_y \forall x:\ \norm{x}=1\}
	\end{align*}
	
\end{proposition}
Soit $X$ un espace de Banach.
\begin{proposition}
	Si $Y$ est un espace de Banach, alors $(B(X,Y),\norm{•})$ est lui même un espace de Banach.
\end{proposition}

Application: $X'$ le dual topologique de $X$:
$φ\in X'$ si $φ\in L(X,\C)$ qui satisfait $\exists C>0\ \forall x\in X$: $|φ(x)|≤C\norm{x}_X$. $\C$ est complet alors par la prop. 3 $X'$ est complet.

\begin{exercise}
	Montrer la proposition 3. Soit $(T_n)_{n\in\N}$ une suite du Cauchy des $B(X,Y)$ il faut montrer $\exists T\in B(X,Y)$ tq $\lim_{n\to∞}\norm{T_n-T}=0.$
\end{exercise}

\section{Adjoint d'un opérateur} % (fold)
\label{sec:adjoint_d_un_operateur}
Soit $\hs$, $\hs'$ deux espaces de Hilbert (séparables).
\begin{proposition}
	Soit $T\in B(\hs,\hs')$, li existe $T^*\in B(\hs',\hs)$ dit opérateur adjoint qui satisfait: $\forall x\in\hs, \forall y\in\hs'$
	$$(Tx|y)=(x|T^*y)$$
\end{proposition}

\begin{example}
	$\hs=\hs'=l^2(\Z)$ $T$ shift adjointe, calculons $T^*$.$\forall u,v\in\hs$, $(Tu|v)=∑_{n\in\Z}Tu(n)•\overline{v(n)}=∑_{n\in\Z}u(n-1)\overline{v(n)}=∑_{e\in\Z}u(e)\overline{v(e+1)}=(u|w)$ avec $w(e)=v(e+1)$. On pose $T^*v=w$.
\end{example}
\begin{proof}
	Dans ces conditions $x\in\hs\mapsto (Tx|y)$ est une forme linéaire sur $\hs$ comme la composition de $T$ et $(•|y)$. De plus on a que $|(Tx|y)|≤\norm{Tx}_{\hs'}\norm{y}_{\hs}≤\norm{T}\norm{x}_{\hs}\norm{y}_{\hs'}$
	$|φ(x)|≤\const\norm{x}_{\hs'}, \const=\norm{T}\norm{y}_{\hs'}$
alors $φ$ est continue (bornée). D'après le Théorème de Riez
$\exists!\in\hs tq φ(x)=(x|z) \forall x\in \hs$. On pose $z=T^*y$, montrons que $T^*\in L(\hs',\hs)$.
Soit $y_1,y_2\in\hs d_1,d_2\in\C$ on calcule $T^*(d_1y_1+d_2y_2)$:\\
* $\forall x\in \hs (x|T^*(d_1y_1+d_2y_2))=(Tx|d_1y_1+d_2y_2)$ (def)

$\bar d_1(Tx|y_1)+\bar d_2(Tx|y_2)=\bar d_1(x|T^*y_1)+\bar d_2(x|T^*y_2)=(x|d_1T^*y_1+d_2T^*y_2)$ $\Rightarrow$ $\forall x\in\hs (x|T^*(d_1y_1+d_2y_2)-d_1T^*y_1-d_2T^*y_2)=0$ $\Rightarrow$ $T^*(d_1y_1+d_2y_2)-d_1T^*y_1-d_2T^*y_2=0_\hs$.

Montrons que $T^*\in B(\hs,\hs') \forall y\in \hs'$.
$\norm{T^*y}^2_\hs=(T^*y|T^*y)_\hs=(T(T^*y)|y)_{\hs'}$ $\Rightarrow$ $\norm{T^*y}^2_{\hs}≤\norm{T(T^*y)}_{\hs'}\norm{y}_{\hs'}≤\norm{T}\norm{T^*y}_\hs\norm{y}_\hs$
$\Rightarrow$ $\norm{T^*y}_\hs≤\norm{T}\norm{y}_{\hs'} \forall y\in \hs'$ tq $T^*y≠0_\hs$. Si $y\in N(T^*)$ on a que $0≤\norm{T}\norm{y}_{\hs'}$ donc $T^*\in B(\hs',\hs)$, $\norm{T^*}≤\norm{T}$.

\underline{Unicité.} $\exists S\in B(\hs',\hs)$ tq $\forall(x,y)\in\hs\times\hs' (Tx|y)=(x|Sy)=(x|T^*y)$ $\Rightarrow$ $\forall x\in\hs\ (x|Sy-T^*y)=0$ $\Rightarrow$ $Sy=T^*y$.
\end{proof}

\begin{example}
	$\hs=\hs'=L^2(\R)$. Soit $f\in C^0(\R)\cap L^∞(\R)$. On définit l'action de $T$ sur $C^∞_0(\R)\ni φ: Tφ(x)=f(x)φ(x)$ $f•φ\in C^∞_0(\R)$: $T:C_0^∞(\R)\mapsto C_0^∞(\R)$. $T$ est linéaire $\Rightarrow$ $fφ\in L^2(\R)$. $T:C_0^∞(\R)\rightarrow L^2(\R)$ est continue.
	$\norm{Tφ}^2=(Tφ|Τφ)=∫_\R f^2(x)φ(x)\bar φ(x)\dd{x}=∫_\R f^2(x)|φ(x)|^2\dd{x}≤\norm{f}^2_∞∫_\R|φ(x)|^2\dd{x}$ $\Rightarrow$ $\norm{Tφ}^2≤\norm{f}_∞^2\norm{φ}^2$. $T$ est continue sur $C_0^∞(\R)\mapsto L^2(\R)$.
	$T$ est uniformément continue car $\norm{Tφ-Tψ}=\norm{T(φ-ψ)}≤\norm{f}\norm{φ-ψ}$. $\norm{f}_∞$ Lipstitz. On utilise que toute applications $T$ uniformément continue sur $D$ et $\bar D=\hs$, admet un prolongement par continuité sur $\hs$ défini comme:
	$\forall φ\in\hs$, $\exists (φ_n)_{n\in\N}$ Suite de $D$ et $\lim_nφ_n=φ$.
	
	On pose $Tφ=\lim_nTφ_n$.
	T est borne et $\norm{T}≤\norm{f}_∞$. $\norm{T_φ}\def \norm{\lim_n Tφ_n}=\lim_n\norm{Tφ_n}$ mais $\norm{Tφ_n}≤\norm{f}_∞\norm{φ_n}$. $\norm{Tf}≤\norm{f}_∞\norm{φ}$.
	Calculons. $T^*$
	$\forallφ, ψ\in \hs (Tφ|ψ)=∫_\R f(x)φ(x)\bar ψ(x)\dd{x}=∫φ(x)\overline{f(x)ψ(x)}\dd{x}=(φ|Tψ)$ $\Rightarrow$ $T^*=T$.	
\end{example}
\begin{remark}
	Dans le preuve de la proposition III21 On peut inverser la rôle de $T$ et $T^*$, alors on montre aussi que $\norm{T^*}≥\norm{T}$ alors $\norm{T^*}=\norm{T}$ (ex)
\end{remark}
\begin{definition}
	Un opérateur $T\in B(\hs)$ est dit auto adjoint si $T=T^*$.
	$T\in B(\hs)$ est dit unitaire si $TºT^*=T^*ºT=\ind_\hs$.
\end{definition}
\begin{remark}
	Si $T=T^* \forall x\in\hs (Tx|x)=(x|Tx)$ $\Rightarrow$ $(Tx|x)=\overline{(Tx|x)}$\Rightarrow$(Tx|x)\in\R$.
\end{remark}
\begin{definition}
	$T=T^*$ est positif si $\forall x\in\hs (Tx|x)≥0$
	$T=T^*$ est définit positif si $\forall x\in\hs$, $x≠0_X (Tx|x)>0$.
	$T$ est défini positif si $T$ est positif et $(Tx|x)=0$ $\Leftrightarrow$ $x=0_X$.
\end{definition}
% section adjoint_d_un_operateur (end)
% section generalites (end)
% chapter operateurs_sur_un_espace_de_hilbert (end)